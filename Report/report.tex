\documentclass[]{article}

\usepackage{amssymb}
\usepackage{amsthm}
\usepackage{amsmath}
\usepackage{textcomp}
\usepackage{stmaryrd}

\newcommand{\stxrule}[3]{\text{#2} & #1 & ::= & #3\\}
\newcommand{\stxrulecont}[1]{& & | & #1 \\}

\newcommand{\TDyn}{\star}

\newcommand{\PUnit}{\mathtt{Unit}}
\newcommand{\PFunction}[2]{#1 \rightarrow #2}
\newcommand{\PProduct}[2]{#1 \times #2}
\newcommand{\PSum}[2]{#1 + #2}
\newcommand{\PRef}[1]{\mathtt{Ref}\;#1}

\newcommand{\tvar}[1]{#1}
\newcommand{\tunit}[0]{\mathtt{unit}}
\newcommand{\tlambda}[4]{\lambda^{\PFunction{#1}{#2}}#3.#4}
\newcommand{\tapp}[2]{(#1 \; #2)}
\newcommand{\tcons}[4]{\mathtt{cons}^{\PProduct{#1}{#2}} \; #3 \; #4}
\newcommand{\tfst}[1]{\mathtt{fst} \; #1}
\newcommand{\tsnd}[1]{\mathtt{snd} \; #1}
\newcommand{\tinl}[3]{\mathtt{inl}^{\PSum{#1}{#2}} \; #3}
\newcommand{\tinr}[3]{\mathtt{inr}^{\PSum{#1}{#2}} \; #3}
\newcommand{\tcase}[3]{\mathtt{case} \; #1 \; #2 \; #3}
\newcommand{\tcast}[4]{#1 \langle #2\Rightarrow^{#3}#4 \rangle}
\newcommand{\tblame}[1]{\mathtt{blame} \; #1}
\newcommand{\tread}[1]{\mathtt{!}\;#1}
\newcommand{\twrite}[2]{#1\;\mathtt{:=}\;#2}

\newcommand{\hcheck}[1]{\text{\textinterrobang}^{#1}}
\newcommand{\hid}[0]{\epsilon}
\newcommand{\mUnit}{\mathtt{Unit}}
\newcommand{\mFunction}[2]{#1 \rightarrow #2}
\newcommand{\mProduct}[2]{#1 \times #2}
\newcommand{\mSum}[2]{#1 + #2}
\newcommand{\mRef}[1]{\mathtt{Ref}\;#1}
\newcommand{\mBot}[2]{{^{#1}}\bot^{#2}}

\newcommand{\ciddyn}{\mathtt{id\star}}
\newcommand{\chmt}[3]{#1 \overset{#2}{\curvearrowright} #3}

%opening
\title{Reversible Hypercoercions}
\author{
	Kuang-Chen \and	Andre
}


\begin{document}

\maketitle

%\begin{abstract}
%
%\end{abstract}

\section{Syntax}

\begin{figure}
	\[
	\begin{array}{lrcl}
	\stxrule{S,T}{Types}{
		\TDyn \mid P
	}
	\stxrule{Q,P}{Pretypes}{
		\PUnit \mid
		\PFunction{T}{T} \mid
		\PProduct{T}{T} \mid
		\PSum{T}{T} \mid
		\PRef{T}
	}
%	\stxrule{s,t}{Terms}{
%		\tvar{x} \mid{}
%		\tunit{} \mid{}
%		\tlambda{T}{T}{x}{t} \mid
%		\tapp{t}{t} \mid
%		\tread{t} \mid
%		\twrite{t}{t}
%	}
%	\stxrulecont{
%		\tcons{T}{T}{t}{t} \mid
%		\tfst{t} \mid
%		\tsnd{t} \mid
%	}
%	\stxrulecont{
%		\tinl{T}{T}{t} \mid
%		\tinl{T}{T}{t} \mid
%		\tcase{t}{t}{t}
%	}
%	\stxrulecont{
%		\tcast{t}{T}{l}{T} \mid
%		\tblame{l}
%	}
\stxrule{c}{Reversible Hypercoercions}{
	\ciddyn \mid
	\chmt{h}{m}{t}
}
\stxrule{m}{Middles}{
\mUnit \mid
\mFunction{c}{c} \mid
\mProduct{c}{c} \mid
\mSum{c}{c} \mid
\mRef{c} \mid
\mBot{l}{l}
}
\stxrule{h,t}{Heads or tails}{
\hid \mid \hcheck{l}
}
	\end{array}
	\]
	
\caption{Reversible Hypercoercions}
\end{figure}

Reversible hypercoercions are not equivalent to hypercoercions as a consequence 
of moving failure to the middles. Consider 
\[
\chmt{\hid}{\mFunction{c_1}{c_2}}{\hcheck{l}}
\;\fatsemi\;
\chmt{\hcheck{l'}}{\mBot{l_1}{l_2}}{\hid}
=
\chmt{\hid}{\mBot{\bullet}{l_2}}{\hid}
\]
What should go into the $\bullet$ depends on whether we want to blame the 
projection $\hcheck{l'}$. The projection is guilty if the left pretype of teh 
second middle is not shallowly consistent with functions, but we have not 
information for making this decision. Here are some potential solutions:
\begin{itemize}
	\item include a type constructor in head
	\item include two type constructors in bottoms.
\end{itemize}

\end{document}
